\selectlanguage{english}
\addcontentsline{toc}{chapter}{References}
\bibliographystyle{amsplain}


\begin{thebibliography}{TIOBE 2021}%size of indentation to match the longest word

%\bibitem[Daw 1979]{Daw 1979} {\sc Dawid, A. P.} (1979). {\em Conditional independence in statistical theory}. J. R. Stat. Soc. B {\bf 41} (1) pp. 1-31.

\bibitem[Ash 1970]{probability} {\sc Ash, R. B.} (1970). {\em Basic probability theory}. Wiley New York.

\bibitem[BLW 1986]{graphs} {\sc Biggs, N., Lloyd, E. K. and Wilson, R. J.} (1986). {\em Graph Theory, 1736-1936}. Clarendon Press.

\bibitem[Pea 1993]{pearl_back_door} {\sc Pearl, J.} (1993). {\em Comment: graphical models, causality and intervention}. Stat. Sci. {\bf 8} (3) pp. 266-269.

\bibitem[Pea 1995]{pearl_do_rules} {\sc Pearl, J.} (1995). {\em Causal Diagrams for Empirical Research}. Biometrika {\bf 82} (4) pp. 669-688.

\bibitem[Pea 2000]{pearl_causality} {\sc Pearl, J.} (2000). {\em Causality: Models, Reasoning, and Inference}. Cambridge University Press.

\bibitem[PGJ 2016]{pearl_inference} {\sc Pearl, J., Glymour, M. and Jewell, N. P.} (2016). {\em Causal inference in statistics: a primer}. Wiley Chichester.

\bibitem[PM 2018]{pearl_why} {\sc Pearl, J. and Mackenzie, D.} (2018). {\em The Book of Why}. Penguin Random House UK.

\bibitem[SP 2006a]{SP_2006a} {\sc Shpitser, I. and Pearl, J.} (2006). {\em Identification of Joint Interventional Distributions in Recursive Semi-Markovian Causal Models}. AAAI Press. pp. 1219-1226.

\bibitem[SP 2006b]{SP_2006b} {\sc Shpitser, I. and Pearl, J.} (2006). {\em Identification of Conditional Interventional Distributions}. AUAI Press. pp. 437-444.

\bibitem[SP 2007]{SP_2007} {\sc Shpitser, I. and Pearl, J.} (2007). {\em What Counterfactuals Can Be Tested}. AUAI Press. pp. 352–359.

\bibitem[Tian 2002]{tian_2002} {\sc Tian, J.} (2002). {\em Studies in Causal Reasoning and Learning}. PhD Dissertation, Department of Computer Science, University of California.

\bibitem[TIOBE 2021]{tiobe} {\sc TIOBE Software BV} (2021). {\em TIOBE Index for June 2021}. \url{https://www.tiobe.com/tiobe-index/}.

\bibitem[TK 2017]{causaleffect_R} {\sc Tikka S. and Karvanen J.} (2017). {\em Identifying Causal Effects with the R Package causaleffect}. J. Stat. Softw. {\bf 76} (12) pp. 1-30.

\bibitem[Ver 1993]{Ver} {\sc Verma, T. S.} (1993). {\em Graphical Aspects of Causal Models}. UCLA Cognitive Systems Laboratory, Technical Report (R-191).

\bibitem[VP 1988]{verma_pearl} {\sc Verma, T. S. and Pearl, J.} (1988). {\em Causal Networks: Semantics and Expressiveness}. UAI.

\end{thebibliography}

\iffalse
\bibitem[Ahl]{ahlfors} {\sc Ahlfors, L. V.} (1979). {\em Complex Analysis}. McGraw-Hill, Inc..

\bibitem[Ale]{alexander} {\sc Alexander, D. S.} (1994). {\em A History of Complex Dynamics: From Schröder to Fatou and Julia}. Springer Vieweg.

\bibitem[Bea]{beardon} {\sc Beardon, A. F.} (1991). {\em Iteration of Rational Functions}. Springer-Verlag.

\bibitem[BF]{branner_fagella} {\sc Branner, B. and Fagella, N.} (2014). {\em Quasiconformal Surgery in Holomorphic Dynamics}. Cambridge University Press.

\bibitem[BFJK1]{BFJK2} {\sc Bara\'nski, K., Fagella, N., Jarque, X. and Karpi\'nska, B.} (2014). {\em On the connectivity of the Julia sets of meromorphic functions}. Invent. Math. {\bf 198} pp. 591-636.

\bibitem[BFJK2]{BFJK1} {\sc Bara\'nski, K., Fagella, N., Jarque, X. and Karpi\'nska, B.} (2018). {\em Connectivity of Julia sets of Newton maps: a unified approach}. Rev. Mat. Iberoam. {\bf 34} pp. 1211-1228.

\bibitem[BT]{bergweiler} {\sc Bergweiler, W. and Terglane, N.} (1996). {\em Weakly repelling fixpoints and the connectivity of wandering domains}. Trans. Amer. Math. Soc. {\bf 348} pp. 1-12.

\bibitem[Buff]{buff} {\sc Buff, X.} (2013). {\em Virtually repelling fixed points}. Publ. Mat. {\bf 47} pp. 195-209.

\bibitem[CG]{carleson_gamelin} {\sc Carleson, L. and Gamelin, T. W.} (1993). {\em Complex Dynamics}. Springer-Verlag.

\bibitem[Con]{conway} {\sc Conway, J. B.} (1978). {\em Functions of One Complex Variable I}. Springer-Verlag.

\bibitem[DH]{DH} {\sc Douady, A. and Hubbard, J. H.} (1985). {\em On the dynamics of polynomial-like mappings}. Ann. Scient. {\bf 18} pp. 287-343.

\bibitem[FJ]{fagella_jarque} {\sc Fagella, N. and Jarque, X.} (2007). {\em Iteración Compleja y Fractales}. Vicens Vives.

\bibitem[FJT1]{FJT1} {\sc Fagella, N., Jarque, X. and Taixés, J.} (2008). {\em On connectivity of Julia sets of transcendental meromorphic maps and weakly repelling fixed points I}. Proc. Lond. Math. {\bf 97} pp. 599-622.

\bibitem[FJT2]{FJT2} {\sc Fagella, N., Jarque, X. and Taixés, J.} (2011). {\em On connectivity of Julia sets of transcendental meromorphic maps and weakly repelling fixed points II}. Fund. Math. {\bf 215} pp. 177-202.

\bibitem[Her]{herman} {\sc Herman, M. R.} (1979). {\em Sur la conjugaison différentiable des difféomorphismes du cercle à des rotations}. Publ. Math. IHÉS {\bf 49} pp. 5-233.

\bibitem[HSS]{hss} {\sc Hubbard, J. H., Schleicher, D. and Sutherland, S.} (2001). {\em How to find all roots of complex polynomials by Newton's method}. Invent. Math. {\bf 146} pp. 1-33.

\bibitem[McM]{mcmullen} {\sc McMullen, C. T.} (1994). {\em Complex Dynamics and Renormalization}. Princeton University Press.

\bibitem[Mei]{meier} {\sc Meier, H. G.} (1989). {\em On the connectedness of the Julia-set for rational functions}. Preprint RWTH Aachen.

\bibitem[MH]{marsden} {\sc Marsden, J. E. and Hoffman, M. J.} (1999). {\em Basic Complex Analysis}. W. H. Freeman.

\bibitem[Mil]{milnor} {\sc Milnor, J.} (2006). {\em Dynamics in One Complex Variable}. Princeton University Press.

%\bibitem[Mun]{munkres} {\sc Munkres, J. R.} (2006). {\em Dynamics in One Complex Variable}. Princeton University Press.

\bibitem[Pom]{pommerenke} {\sc Pommerenke, C.} (1992). {\em Boundary Behaviour of Conformal Maps}. Springer-Verlag.

\bibitem[Prz]{przytycki} {\sc Przytycki, F.} (1989). {\em Remarks on the simple connectedness of basins of sinks for iterations of rational maps}, in: Dynamical systems and ergodic theory, Banach Center Publ. 23, PWN, Warsaw. pp. 229-235.

\bibitem[Shi1]{shishikura1} {\sc Shishikura, M.} (1989). {\em Trees associated with the configuration of Herman rings}, Th. Dynam. Syst. {\bf 9} pp. 207-260.

\bibitem[Shi2]{shishikura2} {\sc Shishikura, M.} (2009). {\em The Connectivity of the Julia Set and Fixed Points}, in: Complex dynamics, A K Peters, Ltd. pp. 257-276. 

\bibitem[Ste]{steinmetz} {\sc Steinmetz, N.} (1993). {\em Rational Iteration}. De Gruyter.

\bibitem[Sul]{sullivan} {\sc Sullivan, D.} (1985). {\em Quasiconformal Homeomorphisms and Dynamics I. Solution of the Fatou-Julia Problem on Wandering Domains}. Ann. of Math. {\bf 122} pp. 401-418.

\bibitem[Tan]{tan_lei} {\sc Tan, L.} (1997). {\em Branched coverings and cubic Newton maps}. Fund. Math. {\bf 154} pp. 207-260.

\bibitem[Why]{whyburn} {\sc Whyburn, G.T.} (1963). {\em Analytic topology}. Amer. Math. Soc. Colloq. Publ., Vol. 28, Amer. Math. Soc., Providence, R.I..


\fi