\chapter[Analysis]{Analysis}

This chapter is the central cornerstone of this dissertation. In it, we will discuss the analysis conducted, starting with a general overview, followed by an explanation of the samples, triggers, and object definitions. We will then discuss the corrections made to the data and simulations to enhance the analysis results. We will also cover the various criteria utilized in event selection and how the signal and background have been modeled. Ultimately, we will present the expected limits for each channel. The chapter concludes by addressing the subsequent steps required prior to data unblinding and the attainment of the final experimental measurement.

The final goal of this thesis is to compute a reasonable upper limit on the branching ratio of the aforementioned Higgs boson decays. Table \ref{tab:Higgs_rare_decays_values} shows the order of magnitude of the branching fractions we ultimately would like to measure, but due to the large hadronic background at the LHC we are targeting an upper limit. 

Because of the time scope of this project only an estimation of the upper limits using leading order monte carlo simulations is going to be computed, although the process would be analogous for real data from the LHC (of course taking into consideration more backgrounds, systematics, etc.). This is a first estimation.

\section{Analysis overview}

The main difference between this analysis and the one studying the three decays in the top half of Table \ref{tab:Higgs_rare_decays} lies in the fact that we are dealing with 3-body decays involving neutral particles, which are more challenging to track compared to charged ones. That is why we will focus most of our attention on accurately recovering the missing neutral particles.

\section{Samples and triggers}

\todo{Explain data, background and signal MC simulation, triggers}

\section{Object definitions}

\todo{Primary vertex, leptons?, jets, missing energy, photons, mesons}

\section{Corrections to data and simulations}

\todo{Pileup reweighting, L1 prefiring corrections, photon scale and resolution, photon mvaid efficiency, Lepton ID reconstruction efficiency and energy scale (?), meson reconstruction (+regression of the pt), triggers scale factors}

\section{Event selection}

\todo{Gluon fusion selection for each channel}

\section{Signal and background modelling}

\todo{signal, background model from MC and data, bias studies}

\section{Results}

\section{Multivariate analysis for final results}

\todo{talk about MVA and what are next steps before unblinding data}

